\documentclass[12pt]{jreport}

\input{jdummy.def}
%卒論用スタイルファイル
\usepackage{jgraduate2003_sjis}
%索引作成用
\usepackage{makeidx}
\usepackage{graphicx}
\usepackage{ascmac}
\usepackage{enumerate}

% 前設定
\date{平成27年2月}
\title{\fontsize{15.5pt}{25pt}\selectfont WLANとZigBeeの共存に向けた\\AA(Access Point-Assisted) CTS-Blockingに関する研究}
\author{佐伯 良光}
\university{九州大学大学院}
\department{システム情報科学府}
\course{修士課程}
\major{情報知能工学専攻}
\subcourse{社会情報システム工学コース}

\begin{document}
%表紙
\maketitle


\chapter{評価}\label{implement}%==============================================

本章では,提案するAA CTS-Blockingシステムの評価について説明する.\\

\section{評価環境}

本システムの有効性を検証する評価として,WLAN通信環境下でのZigBee通信実験を行った.
WLANネットワークが存在する屋内環境として,筆者が所属する九州大学の福田研究室を選択した.
図\ref{fig:5_01}に研究室内の地図を示す.
研究室内に10台のノードを分散して配置し,200\,msごとに全ノードからダミーデータを収集した.
%ダミーデータには送信元アドレス及びシーケンス番号が記録されており,サイズはヘッダを含めて18バイトである.
WLANネットワーク側には5台のWLAN端末を用いて通信させた.
スマートフォン等での動画のストリーミング再生約5\,Mbpsの通信負荷を常時発生させた.
通信負荷値の測定にはパケット解析ツールであるWiresharkを用いた.
評価実験の際には,図\ref{fig:5_02}に示す通りWLANの平均トラフィックが大幅に変化しないことを確認している.

\section{AP選択}

本システムにおいては,RTSの送信先APの選択が干渉回避性能に大きな影響を及ぼすと考えられる.
本稿は初期的評価を通じてAA CTS-Blockingの有効性を検証することを目的としているため,
RTSの送信先APとしてRSSIが最も大きいAPを選択するものとした.
具体的には,制御PCにおいて観測されるAPの中で信号RSSIが最大となるAPをRTSの送信先として選択した.

\section{評価パラメータの決定}

前章前節前項で行った事前実験から,CTS-Block時間及びZigBee通信時間等を決定した.
以下で,詳細を説明する.
本評価実験では,RTS/CTSフレームを用いて30\,ms間のZigBee通信時間を確保した.
ダミーデータの収集ではTDMA方式のアクセス制御を行って各ノード間の通信が衝突しないようにした.
CTSフレームを受信した制御PCからZigBee基地局を用いてデータ送信要求を全ノードに対してブロードキャストする.
データ送信要求を受信すると,各ノードはあらかじめ割り当てられたスロットにおいてダミーデータを送信する.
スロットサイズは2\,msである.
本実験では,RTS/CTSフレームを利用して全10台のノードからダミーデータを収集する動作を1サイクルと定義する.

\section{評価}

評価は,各ノードからのデータ収集を1000サイクル行い,データ収集通信の成功
率を算出した.
比較対象として,
(1)~Normal: 何もせずにZigBee通信を行った場合,
(2)~CTS-Blocking: 制御PCから周囲のWLAN端末へ直接CTSを送信した場合,
(3)~AA CTS-Blocking: 制御PCからAPへRTSを送信した場合
のそれぞれについて実験を行った.


\section{評価結果}\label{sec:exp_result}

図\ref{fig:cts1}に,実験種別(1)~(3)の場合におけるZigBee通信効率を示す.
図\ref{fig:cts1}から,通信効率は(1)に比べ(2),(3)が明らかに向上していることがわかる.
特に(1)と(3)では約8%程度の優位な差が見受けられる.
また,(2)に比しても(3)の方が約5%向上していることがわかる.
これは,制御PCよりも送信電力の高いAPにCTSメッセージを送信させることで,CTS-Blockingの問題点であった隠れ端末問題が改善されていると考えられる.

さらに,図\ref{fig:cts2}に,実験種別(1)~(3)の場合における1サイクルのメッセージ数分布を示す.
これを見ると,(1)では7の辺りにピークがあり,最大値10の度数は低いことがわかる.
しかし,(2)ではそのピークが右へシフトしていることがわかる.
(3)ではその傾向が更に強まっており,最大値の度数も1つのピークとして確認できることから,1サイクルあたりの受信メッセージ数は確実に向上している.

\section{考察}\label{sec:conclu}

本稿では,WLANとZigBeeの共存に向けたAA CTS Blockingを示した.
AA CTS Blockingを用いたデータ収集システムを実装し,実証評価を通
じてAA CTS Blockingの有効性を検証した.
この結果,既存手法よりも通信成功率を5\,\%改善できることを確認した.
現在,通信成功率の更なる向上に向けたRTS送信先APの選択手法を検討している.

\end{document}

%付録
%\appendix

%索引
%printindex

%項
%\subsection{}
%目
%\subsubsection{}
%段落
%\paragraph{}
%小段落
%\subparagraph{}

