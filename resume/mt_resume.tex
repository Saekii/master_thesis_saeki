\documentstyle[a4j]{yokou}
\renewcommand{\baselinestretch}{1}
\newcommand{\synth}{}

\begin{document}
\課題{\hspace*{2zw} 修士論文試問予稿}
\題目{\large\bf WiFiやBluetoothを用いたスマート無線タグシステム}
\副題{\large\bf SNFC(Smart Narrow Field Communication)に関する研究}
\発表者{園田 侑輝}
\教員{福田 晃 教授}
\日付{平成 26 年 2 月 27 日}
\時間{13:20 〜 13:40} %要修正
\場所{システム情報科学研究院 521講義室}    %要修正

\begin{ヘッダ}
  \begin{center}
%\hspace*{2zw}%以下のミニページを右にシフトさせるためのおまじない
    \begin{minipage}{34zw}
    
\vspace{10mm}
    \parindent=1zw 
インターネット上に存在するデジタルコンテンツは年々増加し,煩雑に
存在するそれらコンテンツに対して,実世界から簡単かつ高速に,求め
るコンテンツへアクセスできる方法が必要である.本論文では,スマー
トフォンに搭載されたWiFiやBluetoothといった通信用の電波を電子タグ
として利用することで,実世界からインターネット上へのダイレクトリ
ンクを簡単に実現するスマート無線タグシステムSNFC(Smart Narrow 
Field Communication)を提案する.SNFCでは,通信用の電波に割り当
てられる識別子やMACアドレスをタグとして用いることで,低コスト化
を図るとともに,認識範囲が約10〜20m程度と広くなり,従来のQRコー
ドやNFC(Near Field Communication)と比較して,一度に複数人がタ
グを読み取れるという利点がある.

SNFCを実環境で導入するために,スマートフォン上で動作するタグ読
み取り用アプリケーション及びタグ登録用アプリケーションの開発や
WiFi及びBluetoothの両電波を発信するWiFi-BLE-Tagの開発を行い,九
州大学の学祭で実証実験を行った.その結果として,ユーザの利用デー
タからユーザがいつ,どこで,どのようなコンテンツを取得したか確認
でき,読み取ったタグを時系列で見ることによってユーザの動きをある
程度推測することができた.また,タグから到来する受信信号強度を用
いた優位度を設定し,それを元にユーザアプリケーション内でコンテン
ツをリスト表示することでユーザの求める情報を推奨することが可能で
あると判った.

%また,提案システムの問題点として,タグとなるAPを隣接して設置した
%環境下でのAPの誤判定が挙げられる.この誤判定を低減するために,複
%数のAPから到来する受信信号強度を利用したAPの判定手法を提案し,
%評価実験により全ての測定位置で誤判定が大幅に減少したことを確認した.
\vspace{8mm}

%800文字以内に抑える.22行程度.
\end{minipage}
\end{center}
\end{ヘッダ}
\end{document} 


